We ended up with speedups for inputs of $n>2^{15}$.  This is mostly due to
the overhead of constructing the tree datastructure that we need to traverse.

\subsection{Root mean square error (RMSE)}
Due to the algorithm uses clustering to approximate forces it is expected that
it deviates from the naive implementation, which calculate forces on a
point-to-point basis.\\
Of course, this can be handled by decreasing the threshold function of the cost
of reducing performance, but it is a tradeoff one might have to decide on a
use-case basis.

\begin{Figure}
  \centering
  \begin{tikzpicture}
    \pgfplotsset{set layers}
    \begin{axis}[
      scale only axis,
      width=0.70\textwidth,
      ymin=0, xmin=0, xmax=1,
      legend entries={RMSE},
      legend style={at={(0.60,0.60)},anchor=north west},
      ylabel style={at={(0.04,0.5)}, align = center},
      xlabel={$\theta$},
      ylabel={RMSE},
      grid=major,
      grid style={dashed, gray!180},
      ]
      \addplot [blue] table[x=theta,y=error]{\errorrate};
    \end{axis}
    \begin{axis}[
      scale only axis,
      width=0.70\textwidth,
      axis y line*=right,
      axis x line=none,
      ymin=3000, ymax=4400, xmin=0, xmax=1,
      legend entries={runtime},
      legend style={at={(0.60,0.45)},anchor=north west},
      ylabel style={at={(1.42,0.50)}, align = center},
      ylabel={Runtime ($\mu$s)},
      ]
      \addplot [black] table[x=theta,y=runtime]{\errorrate};
    \end{axis}
  \end{tikzpicture}
  \captionof{figure}{RMSE on a 1024-body simulation with min-max initial
  x-y-z values of $\pm 10000$.}\label{fig:rmse}
\end{Figure}

As observed from \autoref{fig:rmse} we can confirm that our implementation
computes a completely identical result to the naive implementation if we set
$\theta = 0$. Also, as expected our error-rate and speed goes up dependant on
$\theta$. Setting $\theta$ to $0$ is of course meaningless as we get a worse
overall performance.

Importantly, the RMSE never increments over 1 (which is 1 to 10.000 of the
initial simulation space, ie.\ min/max bounds of initial positions for all the
randomly generated masspoints) for a single step. Of course, as one can assume,
then the RMSE will accumulate the further we increase the steps in the
simulation, but the off-by-one error is so small it won't make any noticable
difference if it is not simulated for a prolonged amount of time.


\subsection{Benchmarks}
We ran benchmarks on the provided GPGPU-servers. We specifically
used GPU4 wich has the following hardware specifications:\\
\texttt{GPU: GeForce RTX 2080 Ti}\\
\texttt{CPU: Intel(R) Xeon(R) CPU E5-2650 v2}\\
\texttt{RAM: 119GB available}\footnote{We are unsure about this number, since we was
unable to find the relevant info about memory of the gpu-server. Different terminal
programs claimed different sizes of available random access memory. We also couldn't
find the memory speed.}\\

We solely used the build-in benchmarking tool for futhark, ran from a terminal
as such: \texttt{futhark bench --backend=opencl nbodysim.fut}.

When simulating with a single step on a dataset with $33,554,432$ elements
($2^{25}$) we reach a speedup of $7029.6$. A thing to note is the fact that
$n$ represents number of bodies, with a 3D velocity and position vector and a
mass for each body, and using 32-bit floats for each axis and the mass. When we
have $n = 2^{25}$ we process a total of $939.5$MB. Our faster implementation
simulates through this size in $939.5\mu$s, resulting in a throughput of
$931.3$GB/second.

\begin{Figure}
  \centering
  \begin{tikzpicture}
    \pgfplotsset{set layers}
    \begin{axis}[
      scale only axis,
      width=0.70\textwidth,
      ymode=log,
      xmode=log,
      log basis y={2},
      log basis x={2},
      ymin=0, xmin=1023, xmax=33554432,
      legend entries={Naive, Barnes Hut, $\theta 0.6$},
      legend pos={north west},
      ylabel style={at={(0.04,0.5)}, align = center},
      xlabel={Input size},
      ylabel={Runtime ($\mu$s)},
      grid=major,
      grid style={dashed, gray!180},
      ]
      \addplot table[x=inputsize,y=naive]{\benchone};
      \addplot table[x=inputsize,y=hut]{\benchone};
      %\addplot table[x=inputsize,y=pointsix]{\benchone};
    \end{axis}
    \begin{axis}[
      scale only axis,
      width=0.70\textwidth,
      xmode=log,
      log basis x={2},
      axis y line*=right,
      axis x line=none,
      xmin=1023, xmax=33554432,
      legend entries={Speedup},
      legend style={at={(0.03,0.75)},anchor=north west},
      ylabel style={at={(1.32,0.5)}, align = center},
      ylabel={Speedup},
      ]
      \addplot [black] table[x=inputsize,y expr=\thisrow{naive}/\thisrow{hut}]{\benchone};
    \end{axis}
  \end{tikzpicture}
  \captionof{figure}{Benchmarks from simulating 1 step with a $\theta$ value of
    $0.5$, showing input sizes and
  corresponding time to execute and corresponding speedup between the two
implementations.}\label{fig:eatshit}
\end{Figure}

From the difference in increments between the naive and our implementation of
the BH-algorithm, ie.\ the speeup, in \autoref{fig:eatshit} we can conclude that
we have indeed achieved a better performing algorithm when $n \gtrsim 2^{15}$

\subsection{Compiling and running}
To compile and run a benchmark one needs the following programs
installed on a system:

\begin{itemize}
  \item Futhark
  \item OpenCL headers and library
  \item Your terminal emulator of choice
\end{itemize}

The procedure is as follows, in a terminal emulator: \\
\begin{lstlisting}[language=bash]
  # Generates the datasets we are going to benchmark on
  # This step might take a while
$ ./runtest -g
  # Download the remote packages required by the project
$ futhark pkg sync
  # run the benchmarks
$ futhark bench --backend=opencl nbodysim.fut
\end{lstlisting}

If generating the large datasets takes too long, you should remove them from the
\texttt{runtest} (shell) script and remember to remove them from the tests in
\texttt{nbodysim.fut}.

If one wants to see a visualization of a simulation, one needs to have Make
installed and run \texttt{\$ make \&\& ./nbodysim-gui}. This simulation
represents a small galaxy with orbiting planets.

Take note that the project was tested with futhark git version $098083d$ and has
thrown errors with the lys library on newer versions.

\noindent
\textbf{Controls}:
\begin{itemize}
  \item \texttt{Space}: pause/unpause the simulation
  \item \texttt{n}: Take a single step
  \item \texttt{j/k}: alter simulation speed modifier, the lower the slower
  \item \texttt{F1}: Hide text
  \item \texttt{ESC}: Exit
\end{itemize}
