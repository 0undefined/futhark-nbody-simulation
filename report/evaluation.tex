Though we get not-a-number values in specific cases we can still rely on the
benchmarks we got since we still do the calculations needed in order to complete
the computation, ie.\ we don't stop the simulation prematurely because some
values cannot be used. We ended up with huge speedups for inputs of $n>2^{15}$.
This is mostly due to the overhead of constructing the tree datastructure that
we need to traverse.

\subsection{Root mean square error (RMSE)}
Due to the algorithm uses clustering to approximate forces it is expected that
it deviates from the naive implementation, which calculate forces on a
point-to-point basis.\\
Of course, this can be handled by decreasing the threshold function of the cost
of reducing performance, but it is a tradeoff one might have to decide on a
use-case basis.

\begin{Figure}
  \centering
  \begin{tikzpicture}
    \pgfplotsset{set layers}
    \begin{axis}[
      scale only axis,
      width=0.70\textwidth,
      % ymode=log,
      % xmode=log,
      % log basis y={2},
      % log basis x={2},
      ymin=0, xmin=0, xmax=1,
      legend entries={RMSE},
      legend style={at={(0.6,0.60)},anchor=north west},
      ylabel style={at={(0.04,0.5)}, align = center},
      xlabel={$\theta$},
      ylabel={RMSE},
      grid=major,
      grid style={dashed, gray!180},
      ]
      \addplot table[x=theta,y=error]{\errorrate};
    \end{axis}
    \begin{axis}[
      scale only axis,
      width=0.70\textwidth,
      axis y line*=right,
      axis x line=none,
      ymin=3000, ymax=4200, xmin=0, xmax=1,
      legend entries={runtime},
      legend style={at={(0.60,0.45)},anchor=north west},
      ylabel style={at={(1.42,0.5)}, align = center},
      ylabel={Runtime ($\mu$s)},
      %grid=major,
      %grid style={dashed, gray!180},
      ]
      \addplot [black] table[x=theta,y=runtime]{\errorrate};
    \end{axis}
  \end{tikzpicture}
  \captionof{figure}{RMSE on a 1024-body simulation with min-max initial
  x-y-z values of $\pm 10000$.}\label{fig:eatshit}
\end{Figure}

\subsection{Benchmarks}
We ran benchmarks on the provided GPGPU-servers. We specifically
used GPU4 wich has the following hardware specifications:\\
\texttt{GPU: GeForce RTX 2080 Ti}\\
\texttt{CPU: Intel(R) Xeon(R) CPU E5-2650 v2}\\
\texttt{RAM: 119GB available}\footnote{We are unsure about this number, since we was
unable to find the relevant info about memory of the gpu-server. Different terminal
programs claimed different sizes of available random access memory. We also couldn't
find the memory speed.}\\

We solely used the build-in benchmarking tool for futhark, ran from a terminal
as such: \texttt{futhark bench --backend=opencl nbodysim.fut}.

When simulating with a single step on a dataset with $33,554,432$ elements
($2^{25}$) we reach a speedup of $7029.6$. A thing to note is the fact that
$n$ represents number of bodies, with a 3D velocity and position vector and a
mass for each body, and using 32-bit floats for each axis and the mass. When we
have $n = 2^{25}$ we process a total of $939.5$MB. Our faster implementation
simulates through this size in $939.5\mu$s, resulting in a throughput of
$931.3$GB/second.

\begin{Figure}
  \centering
  \begin{tikzpicture}
    \pgfplotsset{set layers}
    \begin{axis}[
      scale only axis,
      width=0.70\textwidth,
      ymode=log,
      xmode=log,
      log basis y={2},
      log basis x={2},
      ymin=0, xmin=1023, xmax=33554432,
      legend entries={Naive, Barnes Hut},
      legend pos={north west},
      ylabel style={at={(0.04,0.5)}, align = center},
      xlabel={Input size},
      ylabel={Runtime ($\mu$s)},
      grid=major,
      grid style={dashed, gray!180},
      ]
      \addplot table[x=inputsize,y=naive]{\benchone};
      \addplot table[x=inputsize,y=hut]{\benchone};
    \end{axis}
    \begin{axis}[
      scale only axis,
      width=0.70\textwidth,
      xmode=log,
      ymode=log,
      log basis y={2},
      log basis x={2},
      axis y line*=right,
      axis x line=none,
      ymin=0, ymax=7400, xmin=1023, xmax=33554432,
      legend entries={Speedup},
      legend style={at={(0.03,0.75)},anchor=north west},
      ylabel style={at={(1.32,0.5)}, align = center},
      ylabel={Speedup},
      %grid=major,
      %grid style={dashed, gray!180},
      ]
      \addplot [black] table[x=inputsize,y=speedup]{\benchone};
    \end{axis}
  \end{tikzpicture}
  \captionof{figure}{Benchmarks from simulating 1 step with a $\theta$ value of
    $0.5$, showing input sizes and
  corresponding time to execute and corresponding speedup between the two
implementations.}\label{fig:eatshit}
\end{Figure}

From the difference in increments between the naive and our implementation of
the BH-algorithm, ie.\ the speeup, in \autoref{fig:eatshit} we can conclude that
we have indeed achieved a better performing algorithm when $n \gtrsim 2^{15}$

\subsection{Compiling and running}
To compile and run a benchmark one needs the following programs
installed on a system:

\begin{itemize}
  \item Futhark
  \item OpenCL headers and library
  \item Your terminal emulator of choice
\end{itemize}

The procedure is as follows, in a terminal emulator: \\
\begin{lstlisting}[language=bash]
  # Generates the datasets we are going to benchmark on
  # This step might take a while
$ ./runtest -g
  # Download the remote packages required by the project
$ futhark pkg sync
  # run the benchmarks
$ futhark bench --backend=opencl nbodysim.fut
\end{lstlisting}

If generating the large datasets takes too long, you should remove them from the
\texttt{runtest} (shell) script and remember to remove them from the tests in
\texttt{nbodysim.fut}.

If one wants to see a visualization of a simulation, one needs to have Make
installed and run \texttt{\$ make \&\& ./nbodysim-gui}. This simulation
represents a small galaxy with orbiting planets.

\noindent
\textbf{Controls}:
\begin{itemize}
  \item \texttt{Space}: pause/unpause the simulation
  \item \texttt{n}: Take a single step
  \item \texttt{j/k}: alter simulation speed modifier, the lower the slower
  \item \texttt{F1}: Hide text
  \item \texttt{ESC}: Exit
\end{itemize}
