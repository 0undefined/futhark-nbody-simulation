% Her bestemmes dokument klassen, "article" er standard, og den langt de fleste bruger.
\documentclass[11pt,a4paper,titlepage]{article}

\usepackage[utf8]{inputenc}
\usepackage[T1]{fontenc}
\usepackage[english]{babel}
\usepackage[margin=2cm,headheight=13.6cm]{geometry}
\usepackage[final]{pdfpages}
\usepackage{amsmath}
\usepackage{biblatex}
\usepackage{caption}
\usepackage{color}
\usepackage{enumitem}
\usepackage{float}
\usepackage{geometry}
\usepackage{graphicx}
\usepackage[hidelinks]{hyperref}
\usepackage{listings}
\usepackage{mdwlist}
\usepackage{multicol}
\usepackage{pgfkeys}
\usepackage{subfig}
\usepackage{textcomp}
\usepackage{titling}

\usepackage{tikz}
\usetikzlibrary{arrows}
\usepackage{pgfplots}

\renewcommand\columnseprulecolor{\hspace{17pt}}


\DeclareFixedFont{\ttb}{T1}{txtt}{bx}{n}{9} % for bold
\DeclareFixedFont{\ttm}{T1}{txtt}{m}{n}{9}  % for normal

\definecolor{mygreen}{RGB}{0,127,0}
\definecolor{mygray}{RGB}{100,100,100}
\definecolor{mymauve}{RGB}{100,32,255}
\definecolor{lgray}{RGB}{230,230,230}

% Colors:
\definecolor{KU-red}{RGB}{144, 26, 30}
\definecolor{lightblue}{rgb}{0.6,0.6,1.0}
\definecolor{deepblue}{rgb}{0,0,0.5}
\definecolor{deepred}{rgb}{0.6,0,0}
\definecolor{deepgreen}{rgb}{0,0.5,0}
\definecolor{yellow}{rgb}{0.6,0.6,0}
\definecolor{mattegreen}{rgb}{0.45,0.85,0.45}
\definecolor{gray}{rgb}{0.95,0.95,0.95}
\definecolor{mediumgray}{rgb}{0.60,0.65,0.63}
\definecolor{darkgray}{rgb}{0.45,0.45,0.45}
\definecolor{mattered}{rgb}{1.0, 0.13, 0.32}

% Text Coloring:
\newcommand{\dgreen} [1]{{\color{deepgreen}{#1}}}
\newcommand{\green}  [1]{{\color{green}{#1}}}
\newcommand{\yello}  [1]{{\color{yellow}{#1}}}
\newcommand{\blue}   [1]{{\color{blue} {#1}}}
\newcommand{\red}    [1]{{\color{red}  {#1}}}
\newcommand{\grey}   [1]{{\color{gray} {#1}}}
\newcommand{\dgrey}  [1]{{\color{darkgray} {#1}}}

\newcommand\link[2]{\href{#2}{\blue{#1}}}


\lstset{ %
  frame=none,
  backgroundcolor=\color{white},   % choose the background color; you must add \usepackage{color} or \usepackage{xcolor}
  basicstyle=\footnotesize\ttm,        % the size of the fonts that are used for the code
  breakatwhitespace=false,         % sets if automatic breaks should only happen at whitespace
  breaklines=true,                 % sets automatic line breaking
  captionpos=t,                    % sets the caption-position to bottom
  commentstyle=\ttm\color{mediumgray},
  deletekeywords={...},            % if you want to delete keywords from the given language
  escapeinside={\%*}{*)},          % if you want to add LaTeX within your code
  extendedchars=true,              % lets you use non-ASCII characters; for 8-bits encodings only, does not work with UTF-8
  keepspaces=true,                 % keeps spaces in text, useful for keeping indentation of code (possibly needs columns=flexible)
  keywordstyle=\ttm\color{deepblue},
  language=haskell,                % the language of the code
  morekeywords={*,...},            % if you want to add more keywords to the set
  numbers=none,                    % where to put the line-numbers; possible values are (none, left, right)
  numbersep=5pt,                   % how far the line-numbers are from the code
  numberstyle=\ttb\color{darkgray},
  rulecolor=\color{black},         % if not set, the frame-color may be changed on line-breaks within not-black text (e.g. comments (green here))
  showspaces=false,                % show spaces everywhere adding particular underscores; it overrides 'showstringspaces'
  showstringspaces=false,          % underline spaces within strings only
  showtabs=false,                  % show tabs within strings adding particular underscores
  stepnumber=1,                    % the step between two line-numbers. If it's 1, each line will be numbered
  stringstyle=\color{mymauve},     % string literal style
  tabsize=4,                       % sets default tabsize to 2 spaces
  aboveskip=3mm,
  belowskip=3mm,
}

\lstset{%
  morekeywords={%
    map2,
    reduce,
    scan,
    unsafe,
    length,
}}

\usepackage{fancyhdr}
\pagestyle{fancy}
\rhead{Frederik K. Mastratisi \& Oscar Nelin \& Simon Rotendahl}
\usepackage{datetime}
\usepackage{moresize}

\newdateformat{monthyeardate}{%
  \monthname[\THEMONTH] \THEYEAR}

\newcommand{\rulebreak}{%
	\par%
	\vspace{0.9cm}%
    \noindent\rule{4cm}{0.4pt}%
    \vspace{1.2cm}%
    \par%
}

\newcommand{\coverpage}[1]{%
	\pagenumbering{roman}%
	\thispagestyle{empty}%
	\lhead{\textsc{PFP Group Project}}%
    \title{PFP}%
    \author{Frederik Kallestrup Mastratisi \and Oscar Nelin \and Simon Rotendahl}%
    \newgeometry{left=2cm,bottom=0cm,right=2cm,top=0cm}%
	\begin{center}\hspace{0pt}\vfill%
    \uppercase{Datalogisk Institut Københavns Universitet}
	\rulebreak%
    {\Large\textbf{PFP Group Project}}

    \vspace{0.5cm}
    {\HUGE\textbf{\textit{#1}}}

    \vspace{0.5cm}
	\theauthor%
	\par%
	\vspace{0.9cm}%
    \noindent\rule{4cm}{0.4pt}%
    \vspace{0.45cm}
    % If TOC gets to long, uncomment multicols for columns
    %\begin{multicols}{2}
	\tableofcontents
	%\end{multicols}
	\rulebreak%
    \monthyeardate\today\par
    \hspace{0pt}
	\end{center}%
    \vfill
    \hspace{0pt}
	\pagebreak%
    \restoregeometry%
    \pagenumbering{arabic}%
}

\addbibresource{references.bib}
